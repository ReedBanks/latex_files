\documentclass[12pt,a4paper]{report}

%------Packages-----%
\usepackage{amsmath,amsfonts,amssymb,enumerate}
\usepackage[top=1in,bottom=1in,right=1in,left=1.5in]{geometry}
\usepackage{graphicx,titlesec}
\usepackage{hyperref,float}
\usepackage[]{url}

%------Infos-----%
\author{REXFORD ACQUAH}
\title{Research Proposal}
\newcommand{\lecturer}{}
\newcommand{\school}{
	\centering
	\textbf{\large UNIVERSITY OF MINES AND TECHNOLOGY \\TARKWA
	%	\vspace{1cm}	\\FACULTY OF COMPUTER SCIENCE AND ENGINEERING  DEPARTMENT
	}
}

\newcommand{\rpaim}{
	\textbf{\large {NAME: REXFORD ACQUAH \\
		 DEPARTMENT: FACULTY OF COMPUTER AND MATHEMATICAL SCIENCES  \\
		 DURATION: FULL TIME $($4 YEARS$)$ \\
		 SUBMISSION OF PROJECT: AUGUST,2024 \\
		\vspace{2cm}
		INDEX NUMBER: FOE.41.008.014.21
		}}}


%------Body-----%

\begin{document}
	\pagenumbering{arabic}
	% School Info
	\begin{titlepage}
		\school
		\vspace{2cm}
		\begin{center}
			\includegraphics[width=0.6\textwidth]{/home/kali/Documents/latex_files/School//cyber_rp/umat_logo.jpg} % Adjust width as needed
		\end{center}
	%	\vspace{1cm}
		\begin{flushleft}
				\rpaim
		\end{flushleft}

		\vfill

		%	\maketitle
	\end{titlepage}

		%-----STRUCTURE-----%
		%-----PROJECT TITLE -----%
		\begin{flushleft}
			\textbf{Integrating a Secure and Scalable Video Conferencing Service into the University's Online Learning Platform.}
		\end{flushleft}
		%-----PROBLEM STATEMENT-----%
		\textbf{STATEMENT OF PROBLEM} \\
		As online education advances, specifically since the tragic unforeseen event of COVID-19 universities were faced with the issue of providing virtual classrooms.The school currently depends on external platforms like Zoom and Google Meet which does not fully satisfy specific academic needs for the learning environment.This comes along with disjoint user experiences, security risk and accessibility issues.
	%	\vspace{1.5cm}
	\newline
	\\
		The total reliance on third party programs comes with it's own challenges including limitations to app features and difficulty ensuring that all students are able to fully partake in class.Additionally, the lack of	integration causes inefficiencies in sharing resources.
			\newline
		\\In the attempt to tackle the statement of the problem, this project aims to develop and integrate a custom video conferencing feature directly into the university's online learning platform.Through the implementation of advanced technologies, the project would be designed to meet the specific needs of the academic environment, ensuring seamless user experience, enhanced security and privacy, and full accessibility for all students.By building a scalable and reliable in-house solution, the university can better support its online education efforts, providing a more cohesive and secure learning environment.
		\newline
		\\%-----OBJECTIVES OF PROJECT -----%
		\textbf{OBJECTIVES OF THE PROJECT}
		\begin{enumerate}[a]

			\item{
			To design and implement a video meeting service that seamlessly integrates with the university's existing online learning platform, ensuring compatibility and ease of use for students and faculty.}
			\item{
			To build a system capable of handling high traffic and large numbers of concurrent users without compromising on video quality or overall platform performance, ensuring a smooth experience during peak usage times.}

		\end{enumerate}

		%-----METHODS TO BE USED -----%
		\textbf{METHODS TO BE USED}
		\begin{enumerate}[a]
			\item {Literature Review}
			\item{Requirement Analysis and Stakeholder Consultation}
			\item{Software Development and Integration}
			\item {Security Implementation}
			\item{Load Testing and Performance Optimization}
			\item{User Experience Design and Usability Testing}
			\item{Documentation and Training}
		\end{enumerate}
		%-----EXPECTED PUTCOMES -----%
		\newpage
		\textbf{EXPECTED OUTCOMES }\\
		\begin{enumerate}[a]
		\item{
			Designed and implemented video meeting service that seamlessly integrates with the university's existing online learning platform, ensuring compatibility and ease of use for students and faculty.}
		\item{
			Built system capable of handling high traffic and large numbers of concurrent users without compromising on video quality or overall platform performance, ensuring a smooth experience during peak usage times.}


		\end{enumerate}
		%-----FACILITIES TO BE USED -----%
		\textbf{FACILITIES TO BE USED }\\
		\begin{enumerate}[a]
			\item {Personal Laptop}
			\item{The University Of Mines and Technology Library}
			\item {The University's Computer Engineering lab}
			\item{The University Of Mines and Technology Online Database}

		\end{enumerate}
		%-----ORGANIZATION OF THESIS-----%
		\textbf{ORGANIZATION OF THESIS}\\
		The thesis is organized into five chapters.In Chapter 1, it starts with the statement of the problem and ends with the organization of the thesis. In Chapter 2, relevant information relating to the field of study is addressed while Chapter 3 touches on the methods adopted to solve the problem.Chapter 4 discusses and describe the results.Chapter five touches on conclusion of the thesis and making recommendations.
\end{document}
